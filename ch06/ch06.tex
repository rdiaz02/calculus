%%chapter%% 06
\chapter{Improper integrals}\index{integral!improper}\index{improper integral}

\section{Integrating a function that blows up}
When we integrate a function that blows up to infinity at some
point in the interval we're integrating, the result may be either
finite or infinite.

\begin{eg}\label{eg:improper-a}
\egquestion Integrate the function $y=1/\sqrt{x}$ from $x=0$ to $x=1$.

\eganswer The function blows up to infinity at one end of the region
of integration, but let's just try evaluating it, and see what happens.
\begin{align*}
  \int_0^1 x^{-1/2}\der x &= \left.2x^{1/2}\right|_0^1 \\
                         &= 2
\end{align*}

The result turns out to be finite. Intuitively, the reason for this
is that the spike at $x=0$ is very skinny, and gets skinny fast
as we go higher and higher up.
\end{eg}

%%graph%% improper-a-raw func=x**-.5 format=svg xlo=0 xhi=1.1 ylo=0 yhi=10 with=lines xtic_spacing=1 ytic_spacing=5
\smallfig[h]{improper-a}{The integral $\int_0^1 \der x/\sqrt{x}$ is finite.}

\pagebreak

\begin{eg}
\egquestion Integrate the function $y=1/x^2$ from $x=0$ to $x=1$.

\eganswer 
\begin{align*}
  \int_0^1 x^{-2}\der x &= \left.-x^{-1}\right|_0^1 \\
                         &= -1+\frac{1}{0}
\end{align*}

Division by zero is undefined, so the result is undefined.

Another way of putting it, using the hyperreal number system,
is that if we were to integrate
from $\epsilon$ to 1, where $\epsilon$ was an infinitesimal
number, then the result would be $-1+1/\epsilon$, which is
infinite. The smaller we make $\epsilon$, the bigger the
infinite result we get out.

Intuitively, the reason that this integral comes out infinite is
that the spike at $x=0$ is fat, and doesn't get skinny fast enough.
\end{eg}

%%graph%% improper-b-raw func=x**-2 format=svg xlo=0 xhi=1.1 ylo=0 yhi=10 with=lines xtic_spacing=1 ytic_spacing=5
\smallfig[h]{improper-b}{The integral $\int_0^1 \der x/x^2$ is infinite.}

These two examples were examples of improper integrals.

\section{Limits of integration at infinity}

Another type of improper integral is one in which one of the
limits of integration is infinite. The notation
\begin{equation*}
  \int_a^\infty f(x)\der x
\end{equation*}
means the limit of $\int_a^H f(x)\der x$, where $H$ is
made to grow bigger and bigger. Alternatively, we can
think of it as an integral in which the top end of the
interval of integration is an infinite hyperreal number.
A similar interpretation applies when the lower limit is
$-\infty$, or when both limits are infinite.

\begin{eg}
\egquestion Evaluate
\begin{equation*}
  \int_1^\infty x^{-2}\der x 
\end{equation*}

\eganswer
\begin{align*}
  \int_1^H x^{-2}\der x &= \left.-x^{-1}\right|_1^H \\
               &= -\frac{1}{H}+1
\end{align*}
As $H$ gets bigger and bigger, the result gets closer and closer
to 1, so the result of the improper integral is 1.

Note that this is the same graph as in example \ref{eg:improper-a}, but with the $x$ and $y$ axes
interchanged; this shows that the two different types of improper integrals really aren't so different.
\end{eg}

%%graph%% improper-c-raw func=x**-2 format=svg xlo=0 xhi=10 ylo=0 yhi=1 with=lines xtic_spacing=5 ytic_spacing=1
\smallfig[h]{improper-c}{The integral $\int_1^\infty \der x/x^2$ is finite.}

\begin{eg}
\egquestion Newton's law of gravity states that the gravitational force between two objects
is given by $F=Gm_1m_2/r^2$, where $G$ is a constant, $m_1$ and $m_2$ are the objects' masses,
and $r$ is the center-to-center distance between them. Compute the work that must be done
to take an object from the earth's surface, at $r=a$, and remove it to $r=\infty$.

\eganswer
\begin{align*}
  W &= \int_a^\infty \frac{Gm_1m_2}{r^2} \der r \\
    &= Gm_1m_2 \int_a^\infty r^{-2} \der r \\
    &= -Gm_1m_2 \left.r^{-1}\right|_a^\infty  \\
    &= \frac{Gm_1m_2}{a}
\end{align*}

The answer is inversely proportional to $a$. In other words, if we were able to start from
higher up, less work would have to be done.
\end{eg}


\begin{hwsection}

\begin{hw}
Integrate
\begin{equation*}
  \int_0^\infty e^{-x} \der x\eqquad,
\end{equation*}
or show that it diverges.
\end{hw}

\begin{hw}
Integrate
\begin{equation*}
  \int_1^\infty \frac{\der x}{x}\eqquad,
\end{equation*}
or show that it diverges.
\end{hw}

\begin{hw}
Integrate
\begin{equation*}
  \int_0^1 \frac{\der x}{x}\eqquad,
\end{equation*}
or show that it diverges.
\end{hw}

\begin{hwwithsoln}{x-squared-two-to-x-improper}
Integrate
\begin{equation*}
  \int_0^\infty x^2 2^{-x} \der x\eqquad,
\end{equation*}
or show that it diverges.
\end{hwwithsoln}

\begin{hw}
Integrate
\begin{equation*}
  \int_0^\infty e^{-x}\cos x \der x
\end{equation*}
or show that it diverges. (Problem \ref{hw:integration-by-parts-twice-trick} on p.~\pageref{hw:integration-by-parts-twice-trick} suggests
a trick for doing the indefinite integral.)
\end{hw}

\begin{hw}
Prove that
\begin{equation*}
  \int_0^\infty e^{-e^x} \der x
\end{equation*}
converges, but don't evaluate it.
\end{hw}

\begin{hw}\label{hw:laser}
(a) Verify that the probability distribution $\der P/\der x$ given in example \ref{eg:laser} on page \pageref{eg:laser}
is properly normalized.\\
(b) Find the average value of $x$, or show that it diverges.\\
(c) Find the standard deviation of $x$, or show that it diverges.
\end{hw}

\begin{hw}[2]
Prove
\begin{equation*}
  \int_0^\infty e^{-x}x^n\der x = n!\eqquad.
\end{equation*}
\end{hw}

\end{hwsection}
