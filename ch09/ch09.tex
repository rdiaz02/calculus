%%chapter%% 09
\chapter{Iterated integrals}\label{ch:iterated-int}

\section{Integrals inside integrals}\index{integral!iterated}\index{iterated integral}

In various applications, you need to do
integrals stuck inside other integrals. These are known as
iterated integrals, or double integrals, triple integrals, etc. Similar concepts
crop up all the time even when you're not doing calculus, so let's start by
imagining such an example. Suppose you want to count how many squares there are
on a chess board, and you don't know how to multiply eight times eight. You
could start from the upper left, count eight squares across, then continue with
the second row, and so on, until you how counted every square, giving the result
of 64. In slightly more formal mathematical language, we could write the following
recipe: for each row, $r$, from 1 to 8, consider the columns, $c$, from 1 to 8,
and add one to the count for each one of them. Using the sigma notation, this
becomes
\begin{equation*}
  \sum_{r=1}^8 \sum_{c=1}^8 1\eqquad.
\end{equation*}
If you're familiar with computer programming, then you can think of this as
a sum that could be calculated using a loop nested inside another loop.
To evaluate the result (again, assuming we don't know how to multiply, so we
have to use brute force), we can first evaluate the inside sum, which equals
8, giving
\begin{equation*}
  \sum_{r=1}^8 8\eqquad.
\end{equation*}
Notice how the ``dummy'' variable $c$ has disappeared. Finally we do the outside
sum, over $r$, and find the result of 64.

Now imagine doing the same thing with the pixels on a TV screen. The electron
beam sweeps across the screen, painting the pixels in each row, one at a time.
This is really no different than the example of the chess board, but because the
pixels are so small, you normally think of the image on a TV screen as continuous
rather than discrete.
This is the idea of an integral in calculus.
Suppose we want to find the area of a rectangle of width
$a$ and height $b$, and we don't know that we can just multiply to get the
area $ab$. The brute force way to do this is to break up the rectangle into
a grid of infinitesimally small squares, each having width $\der x$ and
height $\der y$, and therefore the infinitesimal area $\der A = \der x \der y$.\index{area!in Cartesian coordinates}
For convenience, we'll imagine that the rectangle's
lower left corner is at the origin. Then the area is given by this integral:
\begin{align*}
  \text{area} &= \int_{y=0}^b \int_{x=0}^a \der A \\
              &= \int_{y=0}^b \int_{x=0}^a \der x \der y 
\end{align*}
Notice how the leftmost integral sign, over $y$, and the rightmost differential, $\der y$,
act like bookends, or the pieces of bread on a sandwich. Inside them, we have
the integral sign that runs over $x$, and the differential $\der x$ that matches
it on the right. Finally, on the innermost layer, we'd normally
have the thing we're integrating, but here's it's 1, so I've omitted it. Writing the
lower limits of the integrals with $x=$ and $y=$ helps to keep it straight which
integral goes with with differential. The result is
\begin{align*}
  \text{area} &= \int_{y=0}^b \int_{x=0}^a \der A \\
              &= \int_{y=0}^b \int_{x=0}^a \der x \der y \\
              &= \int_{y=0}^b \left(\int_{x=0}^a \der x\right) \der y \\
              &= \int_{y=0}^b a \der y \\
              &= a \int_{y=0}^b \der y \\
              &= ab\eqquad.
\end{align*}

\begin{eg}[Area of a triangle]
\egquestion Find the area of a 45-45-90 right triangle having legs $a$.

\eganswer Let the triangle's hypotenuse run from the origin to the point $(a,a)$,
and let its legs run from the origin to $(0,a)$, and then to $(a,a)$. In other
words, the triangle sits on top of its hypotenuse. Then the integral can be set
up the same way as the one before, but for a particular value of $y$, values of
$x$ only run from 0 (on the $y$ axis) to $y$ (on the hypotenuse). We then have
\begin{align*}
  \text{area} &= \int_{y=0}^a \int_{x=0}^y \der A \\
              &= \int_{y=0}^a \int_{x=0}^y \der x \der y \\
              &= \int_{y=0}^a \left(\int_{x=0}^y \der x\right) \der y \\
              &= \int_{y=0}^a y \der y \\
              &= \frac{1}{2}a^2
\end{align*}
Note that in this example, because the upper end of the $x$ values depends
on the value of $y$, it makes a difference which order we do the integrals
in. The $x$ integral has to be on the inside, and we have to do it first.
\end{eg}

\begin{eg}[Volume of a cube]
\egquestion Find the volume of a cube with sides of length $a$.

\eganswer This is a three-dimensional example, so we'll have integrals
nested three deep, and the thing we're integrating is the volume
$\der V = \der x \der y \der z$.

\begin{align*}
  \text{volume} &= \int_{z=0}^a \int_{y=0}^a \int_{x=0}^a \der V \\
              &= \int_{z=0}^a \int_{y=0}^a \int_{x=0}^a \der x \der y \der z \\
              &= \int_{z=0}^a \int_{y=0}^a a \der y \der z \\
              &= a \int_{z=0}^a \int_{y=0}^a \der y \der z \\
              &= a \int_{z=0}^a a \der z \\
              &= a^2 \int_{z=0}^a \der z \\
              &= a^3
\end{align*}
\end{eg}

\begin{eg}[Area of a circle]
\egquestion Find the area of a circle.

\eganswer To make it easy, let's find the area of a semicircle and then double it.
Let the circle's radius be $r$, and let it be centered on the origin and bounded
below by the $x$ axis. Then the curved edge is given by the equation $R^2=x^2+y^2$,
or $y=\sqrt{R^2-x^2}$. Since the $y$ integral's limit depends on $x$, the $x$
integral has to be on the outside.
The area is
\begin{align*}
  \text{area} &= \int_{x=-R}^r \int_{y=0}^{\sqrt{R^2-x^2}} \der y \der x\\
              &= \int_{x=-R}^r \sqrt{R^2-x^2} \der x\\
              &= r \int_{x=-R}^r \sqrt{1-(x/R)^2} \der x\eqquad.
\intertext{Substituting $u=x/R$,}
  \text{area} & = R^2 \int_{u=-1}^1 \sqrt{1-u^2} \der u 
\end{align*}
The definite integral equals $\pi$, as you can
find using a trig substitution or simply by looking it
up in a table, and the result is, as expected, $\pi R^2/2$ for the area of
the semicircle. Doubling it, we find the expected result of $\pi R^2$ for
a full circle.
\end{eg}


\section{Applications}

Up until now, the integrand of the innermost integral has always been 1, so
we really could have done all the double integrals as single integrals. The
following example is one in which you really need to do iterated integrals.

\smallfig{blondin}{The famous tightrope walker Charles Blondin uses a long pole
for its large moment of inertia.}

\begin{eg}[Moments of inertia]\index{moment of inertia}
The moment of inertia\index{moment of inertia} is a measure of how difficult
it is to start an object rotating (or stop it). For example, tightrope walkers
carry long poles because they want something with a big moment of inertia.
The moment of inertia is defined by $I=\int R^2 \der m$, where $\der m$ is
the mass of an infinitesimally small portion of the object, and $R$ is the
distance from the axis of rotation.

To start with, let's do an example that doesn't require iterated integrals.
Let's calculate the moment of inertia of a thin rod of
mass $M$ and length $L$ about a line perpendicular to the rod
and passing through its center.

\begin{align*}
	I	&= \int R^2 \der m \\
		&= \int_{-L/2}^{L/2} x^2\:\frac{M}{L}\der x \\
\intertext{[$r=|x|$, so $R^2=x^2$]}
		&= \frac{1}{12}ML^2
\end{align*}

Now let's do one that requires iterated integrals:
the moment of inertia of a cube of side $b$,
for rotation about an axis that passes through its center
and is parallel to four of its faces.

Let the origin be at the center of the cube, and
let $x$ be the rotation axis.
\begin{align*}
	I	&=  \int R^2 \der m \\
		&= \rho \int R^2 \der V \\
		&= \rho \int_{b/2}^{b/2} \int_{b/2}^{b/2} \int_{b/2}^{b/2} \left(y^2+z^2\right)
					 \der x\der y\der z \\
		&= \rho b \int_{b/2}^{b/2} \int_{b/2}^{b/2}  \left(y^2+z^2\right)
					 \der y \der z 
\end{align*}
The fact that the last step is a trivial integral results
from the symmetry of the problem. The integrand of the
remaining double integral breaks down into two terms, each
of which depends on only one of the variables, so we break
it into two integrals,
\begin{align*}
		I =& \rho b \int_{b/2}^{b/2} \int_{b/2}^{b/2}  y^2 \der y\der z \\
			&+ \rho b \int_{b/2}^{b/2} \int_{b/2}^{b/2}  z^2 \der y\der z
\end{align*}
which we know have identical results. We therefore only need
to evaluate one of them and double the result:
\begin{align*}
	I	&= 2\rho b \int_{b/2}^{b/2} \int_{b/2}^{b/2}  z^2 \der y\der\: z \\
		&= 2 \rho b^2 \int_{b/2}^{b/2} z^2 \der z \\
		&= \frac{1}{6} \rho b^5 \\
		&= \frac{1}{6} M b^2
\end{align*}
\end{eg}

\pagebreak

\section{Polar coordinates}\index{polar coordinates}\index{coordinates!polar}\index{area!in polar coordinates}

\smallfig{descartes}{Ren\'{e} Descartes (1596-1650)}\index{Descartes, R\'{e}ne}

Philosopher and mathematician Ren\'{e} Descartes originated the idea of describing plane
geometry using $(x,y)$ coordinates measured from a pair of perpendicular coordinate
axes. These rectangular coordinates are known as Cartesian coordinates, in his honor.\index{Cartesian coordinates}\index{coordinates!Cartesian}

\fig{polar-coordinates}{Polar coordinates.}

As a logical extension of Descartes' idea, one can find different ways of defining coordinates on
the plane, such as the polar coordinates in figure \figref{polar-coordinates}. In polar coordinates,
the differential of area, figure \figref{da-polar} can be written as $\der a=R \der R\der \phi$. The idea
is that since $\der R$ and $\der \phi$ are infinitesimally small, the shaded area in the figure is very
nearly a rectangle, measuring $\der R$ is one dimension and $R\der \phi$ in the other. (The latter follows
from the definition of radian measure.)

\smallfig{da-polar}{The differential of area in polar coordinates}

\begin{eg}
\egquestion A disk has mass $M$ and radius $b$. Find its moment of inertia for rotation about the axis
passing perpendicularly through its center.

\eganswer
\begin{align*}
  I &= \int R^2 \der M \\
    &= \int R^2 \frac{\der M}{\der a}\der a \\
    &= \int R^2 \frac{M}{\pi b^2}\der a \\
    &= \frac{M}{\pi b^2}\int_{R=0}^b \int_{\phi=0}^{2\pi} R^2 \cdot R \der \phi\der R \\
    &= \frac{M}{\pi b^2}\int_{R=0}^b R^3 \int_{\phi=0}^{2\pi} \der \phi\der R \\
    &= \frac{2 M}{ b^2}\int_{R=0}^b R^3 \der R \\
    &= \frac{M b^4}{2}
\end{align*}
\end{eg}

%%graph%% poisson-trick func=exp(-x*x) format=eps xlo=-3 xhi=3 ylo=0 yhi=1.1 with=lines xtic_spacing=1 ytic_spacing=1
\smallfig{poisson-trick}{The function $e^{-x^2}$, example \ref{eg:poisson-trick}.}

\begin{eg}\label{eg:poisson-trick}
In statistics, the standard ``bell curve'' (also known as the normal distribution or Gaussian)
is shaped like $e^{-x^2}$. An area under this curve is proportional to the probability that $x$
lies within a certain range. To fix the constant of proportionality, we need to evaluate
\begin{equation*}
  I = \int_{-\infty}^\infty e^{-x^2} \der x\eqquad,
\end{equation*}
which corresponds to a probability of 1.
As discussed on p.~\pageref{impossible-integrals}, the corresponding indefinite integral can't be
done in closed form. The definite integral from $-\infty$ to $+\infty$, however, can be evaluated
by the following devious trick due to Poisson. We first write $I^2$ as a product of two copies
of the integral.
\begin{equation*}
  I^2 = \left(\int_{-\infty}^\infty e^{-x^2} \der x\right) \left(\int_{-\infty}^\infty e^{-x^2} \der x\right)
\end{equation*}
Since the variable of integration $x$ is a ``dummy'' variable, we can choose it to be any letter of the
alphabet. Let's change the second one to $y$:
\begin{equation*}
  I^2 = \left(\int_{-\infty}^\infty e^{-x^2} \der x\right) \left(\int_{-\infty}^\infty e^{-y^2} \der y\right)
\end{equation*}
This is in principle a pointless and trivial change, but it suggests visualizing the right-hand side in
the Cartesian plane, and considering it as the integral of a single function that depends on both $x$ and $y$:
\begin{align*}
  I^2 = \int_{-\infty}^\infty  \int_{-\infty}^\infty \left( e^{-y^2}e^{-x^2}\right) \der x \der y
\end{align*}
Switching to polar coordinates, we have
\begin{align*}
  I^2 &=  \int_0^{2\pi} \int_0^\infty e^{-R^2} R \der R \der \phi \\
      &=  2\pi \int_0^\infty e^{-R^2} R \der R\eqquad,
\end{align*}
which can be done using the substitution $u=R^2$, $\der u=2R \der R$:
\begin{align*}
  I^2 &=  2\pi \int_0^\infty e^{-u} (\der u/2) \\
      &= \pi \\
  I &= \sqrt{\pi}
\end{align*}
\end{eg}

\section{Spherical and cylindrical coordinates}\index{cylindrical coordinates}\index{coordinates!cylindrical}

In cylindrical coordinates $(R,\phi,z)$,
$z$ measures distance
along the axis, $R$ measures distance from the axis, and $\phi$
is an angle that wraps around the axis.
%
\fig{cylindrical-coordinates}{Cylindrical coordinates.}

The differential of volume in cylindrical coordinates can be\index{volume!in cylindrical coordinates}
written as $\der v = R \der R \der z \der \phi$. This follows from adding a third dimension,
along the $z$ axis, to the rectangle in figure \figref{da-polar}.

\begin{eg}
\egquestion Show that the expression for $\der v$ has the right units.

\eganswer Angles are unitless, since the definition of radian measure involves a distance
divided by a distance. Therefore the only factors in the expression that have units
are $R$, $\der R$, and $\der z$. If these three factors are measured, say, in meters, then
their product has units of cubic meters, which is correct for a volume.
\end{eg}

\begin{eg}
\egquestion Find the volume of a cone whose height is $h$ and whose base has radius $b$.

\eganswer Let's plan on putting the $z$ integral on the outside of the sandwich. That means
we need to express the radius $r_{max}$ of the cone in terms of $z$. This comes out nice and
simple if we imagine the cone upside down, with its tip at the origin.
Then since we have $r_{max}(z=0)=0$, and $r_{max}(h)=b$, evidently $r_{max}=zb/h$.

\begin{align*}
  v &= \int \der v \\
    &= \int_{z=0}^h \int_{r=0}^{zb/h} \int_{\phi=0}^{2\pi} R \der\phi \der R \der z \\
    &= 2\pi \int_{z=0}^h \int_{r=0}^{zb/h} R \der R \der z \\
    &= 2\pi \int_{z=0}^h (zb/h)^2/2 \der z \\
    &= \pi (b/h)^2 \int_{z=0}^h z^2 \der z \\
    &= \frac{\pi b^2 h}{3}
\end{align*}

As a check, we note that the answer has units of volume. This is the classical result, known
by the ancient Egyptians, that a cone has one third the volume of its enclosing cylinder.
\end{eg}

In spherical coordinates\index{spherical coordinates}\index{coordinates!spherical}
$(r,\theta,\phi)$, the coordinate
$r$ measures the distance from the origin, and $\theta$ and $\phi$ are analogous to
latitude and longitude, except that $\theta$ is measured down from the pole rather than
from the equator.

\fig{spherical-coordinates}{Spherical coordinates.}

The differential of volume in spherical coordinates is $\der v = r^2\sin\theta \der r\der\theta\der\phi$.\index{volume!in spherical coordinates}

\begin{eg}
\egquestion Find the volume of a sphere.

\eganswer
\begin{align*}
  v &= \int \der v \\
    &= \int_{\theta=0}^\pi \int_{r=0}^{r=b} \int_{\phi=0}^{2\pi} r^2\sin\theta \der\phi \der r\der\theta \\
    &= 2\pi \int_{\theta=0}^\pi \int_{r=0}^{r=b}  r^2\sin\theta \der r\der\theta \\
    &= 2\pi\cdot\frac{b^3}{3} \int_{\theta=0}^\pi \sin\theta \der\theta \\
    &= \frac{4\pi b^3}{3}
\end{align*}
\end{eg}

\begin{hwsection}

\begin{hw}\label{hw:snail}
Pascal's snail (named after \'{E}tienne Pascal, father of Blaise Pascal)
is the shape shown in the figure, defined by $R=b(1+\cos\theta)$ in polar coordinates.\\
(a) Make a rough visual estimate of its area from the figure.\\
(b) Find its area exactly, and check against your result from part a.\\
% ans: 3pi/2
(c) Show that your answer has the right units.\thompson
\end{hw}

\fig{hw-snail}{Problem \ref{hw:snail}: Pascal's snail with $b=1$.}

\begin{hw}
A cone with a curved base is defined by $r\le b$ and $\theta\le \pi/4$ in spherical
coordinates.\\
(a) Find its volume.\\
(b) Show that your answer has the right units.
\end{hw}

\begin{hw}
Find the moment of inertia of a sphere for rotation about an axis passing through its center.
\end{hw}

\begin{hw}
A jump-rope swinging in circles has the shape of a sine function. Find the volume enclosed by
the swinging rope, in terms of the radius $b$ of the circle at the rope's fattest point, and
the straight-line distance $\ell$ between the ends.
\end{hw}

\begin{hw}
A curvy-sided cone is defined in cylindrical coordinates by $0\le z\le h$ and
$R\le kz^2$. (a) What units are implied for the constant $k$? (b) Find the volume
of the shape. (c) Check that your answer to b has the right units.
\end{hw}

\begin{hw}\label{hw:nuclear-shape}
The discovery of nuclear fission\index{nucleus}\index{fission}\index{liquid drop model}
was originally explained by modeling the atomic nucleus as
a drop of liquid. Like a water balloon, the drop could spin or vibrate, and if the motion became
sufficiently violent, the drop could split in half --- undergo fission. It was later learned that
even the nuclei in matter under ordinary conditions are often not
spherical but deformed, typically with an elongated ellipsoidal shape like an American football.
One simple way of describing such a shape is with the equation
\begin{align*}
  r \le b[1+c(\cos^2\theta-k)]\eqquad,
\end{align*}
where $c=0$ for a sphere, $c>0$ for an elongated shape, and $c<0$ for a flattened one.
Usually for nuclei in ordinary matter, $c$ ranges from about 0 to $+0.2$. The constant
$k$ is introduced because without it, a change in $c$ would entail not just a change in
the shape of the nucleus, but a change in its volume as well. Observations show, on the
contrary, that the nuclear fluid is highly incompressible, just like ordinary water, so the
volume of the nucleus is not expected to change significantly, even in violent processes
like fission. Calculate the volume of the nucleus, throwing away terms of order $c^2$
or higher, and show that $k=1/3$ is required in order to keep the volume constant.
\end{hw}
% c=.95beta, beta is known as Hill-Wheeler, see Ring & Schuck p.34

\begin{hw}[2]
This problem is a continuation of problem \ref{hw:nuclear-shape}, and assumes the result of that problem
is already known.
The nucleus $^{168}\zu{Er}$ has the type of elongated ellipsoidal shape described in that
problem, with $c>0$. Its mass is
$2.8\times10^{-25}$ kg, it is observed to have a moment of inertia of $2.62\times10^{-54}\ \kgunit\unitdot\munit^2$
for end-over-end rotation, and its shape is believed to be described by $b\approx 6\times10^{-15}\ \munit$ and $c\approx 0.2$.
Assuming that it rotated rigidly, the usual equation for the moment of inertia could be applicable, but
it may rotate more like a water balloon, in which case its moment of inertia would be significantly less because
not all the mass would actually flow. Test which type of rotation it is by calculating its moment of inertia for end-over-end
rotation and comparing with the observed moment of inertia.
\end{hw}

\begin{hw}
Von K\'arm\'an found empirically that when a fluid flows turbulently through a cylindrical pipe, the
velocity of flow $v$ varies according to the ``1/7 power law,'' $v/v_\zu{o}=(1-r/R)^{1/7}$, where
$v_\zu{o}$ is the velocity at the center of the pipe, $R$ is the radius of the pipe, and $r$ is the
distance from the axis. Find the average velocity at which water is transported through the pipe.
\end{hw}

\end{hwsection}
