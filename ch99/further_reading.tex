%%chapter%% 99
\chapter{References and Further Reading}\label{ch:further-reading}

\subsection{Further Reading}

The amount of high-quality material on elementary calculus available for free online
these days is an embarrassment of riches, so most of my suggestions for reading are
online. I'll refer to books in this section only by the surname of the first author;
the references section below tells you where to find the book online or in print.

The reader who
wants to learn more about the hyperreal system might want to start
with Stroyan and the Mathforum.org article.
For more depth, one could next read the relevant parts of Keisler.
The standard (difficult) treatise on the subject is Robinson.

Given sufficient ingenuity, it's possible to develop a surprisingly large amount of
the machinery of calculus without using limits \emph{or} infinitesimals.
Two examples of such treatments that are freely available online are
Marsden and Livshits. Marsden gives a geometrical definition
of the derivative similar to the one used in ch. 1 of this book, but in my opinion
his efforts to develop a sufficient body of techniques without limits or infinitesimals
end up bogging down in complicated formulations that have the same flavor as the
Weierstrass definition of the limit and are just as complicated. Livshits
treats differentiation of rational functions as division of functions.

Tall gives an interesting construction of a number system that is smaller than the hyperreals,
but easier to construct explicitly, and sufficient to handle calculus involving analytic functions.

\subsection{References}\label{references}

Keisler, J., \emph{Elementary Calculus: An Approach Using Infinitesimals}, \verb@www.math.wisc.edu/~keisler/calc.html@

Livshits, Michael, \verb@mathfoolery.org/calculus.html@

Marsden and Weinstein, \emph{Calculus Unlimited}, \verb@www.cds.caltech.edu/~marsden/books/Calculus_Unlimited.html@

Mathforum.org, \emph{Nonstandard Analysis and the Hyperreals}, \verb@http://mathforum.org/dr.math/faq/analysis_hyperreals.html@

Robinson, A., \emph{Non-Standard Analysis}, Princeton University Press

Stroyan, K., \emph{A Brief Introduction to Infinitesimal Calculus}, \verb@www.math.uiowa.edu/~stroyan/InfsmlCalculus/InfsmlCalc.htm@

Tall, D., \emph{Looking at graphs through infinitesimal microscopes, windows and telescopes}, Mathematical Gazette, 64, 22-49, 
\verb@http://www.warwick.ac.uk/staff/David.Tall/downloads.html@
